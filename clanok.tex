% Metódy inžinierskej práce

\documentclass[10pt,oneside,slovak,a4paper]{article}

\usepackage[slovak]{babel}
%\usepackage[T1]{fontenc}
\usepackage[IL2]{fontenc} % lepšia sadzba písmena Ľ než v T1
\usepackage[utf8]{inputenc}
\usepackage{graphicx}
\usepackage{url} % príkaz \url na formátovanie URL
\usepackage{hyperref} % odkazy v texte budú aktívne (pri niektorých triedach dokumentov spôsobuje posun textu)

\usepackage{cite}
%\usepackage{times}

\pagestyle{headings}

\title{Gamifikácia ako obojstranná cepeľ\thanks{Semestrálny projekt v predmete Metódy inžinierskej práce, ak. rok 2022/23, vedenie: Igor Stupavský}} 

\author{Robert Razgyel\\[2pt]
	{\small Slovenská technická univerzita v Bratislave}\\
	{\small Fakulta informatiky a informačných technológií}\\
	{\small \texttt{xrazgyel@stuba.sk}}
	}

\date{\small 11. október 2022} % upravte



\begin{document}

\maketitle

\begin{abstract}
Gamifikácia je pojem na ktorý je zameraných mnoho rôznych štúdií a článkov. Je mnohými ponímaná iba v dobrom svetle, znamenajúc, že gamifikácia je bezchybná forma vzdelávania. Nie je to ale tak, článkov a štúdií o jej negatívnej stránke nie je dostatok. Na základe tohto predpokladu sa zameriavam na otátky "Aké sú pozitívne účinky gamifikácie?", "Aké sú negatívne účinky gamifikácie?" a "Ako minimalizovať negatíva gamifikácie?". 
\end{abstract}



\section{Úvod}

Budeme sa venovať pozitívnym a negatívnym efektom gamifikácie. Po krátkom predstavení pojmu gamifikácia, bude uvedených niekoľkých príkladov na dobrú či zlú gamifikáciu a následne sa presunieme na možnosti minimalizovania negatívnych efektov gamifikácie.

Pojem gamifikácia je bližšie opýsaný v časti ~\ref{CoJeGamifikacia}.
Pozitíne dôsledky gamifikácie sú bližšie rozobraté v časti ~\ref{PozitivnaGamifikacia} a~\ref{PozitivnaPriklady}.
Negatívne dôsledky použitia gamifikácie sú blžšie rozobraté v časti ~\ref{NegativnaGamifikacia} a~\ref{NegativnePriklady}.
Spôsob minimalizovania negatívnych efektov gamifikácie~\ref{MinimalizovanieNegatGamif}.
Záverečné poznámky prináša časť~\ref{Zaver}.

\section{Čo je to gamifikácia?} \label{CoJeGamifikacia}

Gamifikácia je istá technológia, alebo spôsob, ktorej cieľom je zaujať používateľov takým spôsobom aby vzdelávanie sa nevyzeralo ako vzdelávanie sa. V respektíve, aby sa učenie čo najviac podobalo na hranie hry. Týmto spôsobom sa mení vzťah študenta ku vzdelávaniu, zväčša k lepšiemu. Toto je podstata gamifikácia a dôvod jej popularity.

\section{Pozitívne stránky gamifikácie} \label{PozitivnaGamifikacia}

Je potrebné vedieť kedy by sa mala gamifikácia použiť. Gamifikáciu je najlepšie použiť v prípadoch, keď sú dané činnosti nudné, či až priam nepríjemné. Spôsob akým sa toto v reálnom svete dá implementovať je napríklad použitie jednoduchých odmien za každodenné činnosti. Takouto odmenou môže byť čokoľvek od sladkostí až po materiálne odmeny, ako napríklad odznak za dobre vykonanú prácu. Tento spôsob odmeňovania je často používaný na vprvých ročníkoch základných škôl, kde sú deti známkované napríklad pečiatkou zvieratka. Je to efektívny spôsob, ako spríjemniť vypracovávanie domácich úloh. Toto funguje, lebo človek ktorý vie, že za dobre vykonanú prácu dostane odmenu. Bude sa sústreďovať na danú odmenu a nie na fakt toho že musí vykonávať činnosť ktorú nechce.

\subsection{Konkrétne prípady pozitív gamifikácie} \label{PozitivnaPriklady}
Pre uzrejmenie a poukázanie na pozitíva gamifikácie bude vymenovaných niekoľko príkladov na jej použitie.

Ako bolo spomenuté v časti ~\ref{PozitivnaGamifikacia}, odmenenie detí v školách je častá praktika. Následovať budú dve veľmi dôležité pozitíva gamifikácie.
	\begin{itemize}
	\item Motivácia pre deti do učenia, ktorá prispieva ich budúcnosti
	\item Pochopenie potreby vykonávania povynností
	\end{itemize}
Použitie kvízov na interaktívne vzdelávanie.
\begin{itemize}
\item Odstránenie alebo minimalizovanie vzhľadu tradičnej výučby
\end{itemize}



\section{Negatívne stránky gamifikácie} \label{NegativnaGamifikacia}

Tu opíšem ako môže gamifikácia negatívne ovplyvniť schopnosť učenia sa atď.
Táto časť bude viacej teoretická a bude sa zameriavať na vedeckú a psychyckú stránku negatív gamifikácie

Môže sa mnohým zdať, že negatíva gamifikácie sú až takmer neexistujúce, ale bolo dokázané, že to tak \emph{nie je TUTO CITOVAT}~\cite{Czarnecki:Staged, Czarnecki:Progress}. Napriek tomu, aj dnes na internete narazíme na mnohé pochybné názory \cite{PLP-Framework}.


\subsection{Konkrétne prípady negatív gamifikácie} \label{NegativnePriklady}

Príklady negatív gamifikácie:

\begin{itemize}
\item XXXXXXXXX
\item XXXXXXXXX
	\begin{itemize}
	\item XXXXXXXXX
	\item XXXXXXXXX
	\end{itemize}
\end{itemize}


\section{Ako minimalizovať negatíva gamifikácie?} \label{MinimalizovanieNegatGamif}

\paragraph{Negatív gamifikácie sa nedá naplno zbaviť.}
To však neznamená že sa by sa jej účinky nedali minimalizovať.


\section{Záver} \label{Zaver} % prípadne iný variant názvu



%\acknowledgement{Ak niekomu chcete poďakovať\ldots}


% týmto sa generuje zoznam literatúry z obsahu súboru literatura.bib podľa toho, na čo sa v článku odkazujete
\bibliography{literatura}
\bibliographystyle{plain} % prípadne alpha, abbrv alebo hociktorý iný
\end{document}