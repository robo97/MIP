% Metódy inžinierskej práce

\documentclass[10pt,twoside,slovak,a4paper]{article}

\usepackage[slovak]{babel}
%\usepackage[T1]{fontenc}
\usepackage[IL2]{fontenc} % lepšia sadzba písmena Ľ než v T1
\usepackage[utf8]{inputenc}
\usepackage{graphicx}
\usepackage{url} % príkaz \url na formátovanie URL
\usepackage{hyperref} % odkazy v texte budú aktívne (pri niektorých triedach dokumentov spôsobuje posun textu)

\usepackage{cite}
%\usepackage{times}

\pagestyle{headings}

\title{Gamifikácia ako obojstranná cepeľ\thanks{Semestrálny projekt v predmete Metódy inžinierskej práce, ak. rok 2022/23, vedenie: Igor Stupavský}} % meno a priezvisko vyučujúceho na cvičeniach

\author{Robert Razgyel\\[2pt]
	{\small Slovenská technická univerzita v Bratislave}\\
	{\small Fakulta informatiky a informačných technológií}\\
	{\small \texttt{xrazgyel@stuba.sk}}
	}

\date{\small 11. október 2022} % upravte



\begin{document}

\maketitle

\begin{abstract}
Gamifikácia je pojem na ktorý je zameraných mnoho rôznych štúdií a článkov. Je mnohými ponímaná iba v dobrom svetle, znamenajúc, že gamifikácia je bezchybná forma vzdelávania. Nie je to ale tak, článkov a štúdií o jej negatívnej stránke nie je dostatok. Na základe tohto predpokladu sa zameriavam na otátky "Aké sú pozitívne účinky gamifikácie?", "Aké sú negatívne účinky gamifikácie?" a "Ako minimalizovať negatíva gamifikácie?". 
\end{abstract}



\section{Úvod}

Budeme sa venovať pozitívnym a negatívnym efektom gamifikácie. Po uvedení niekoľkých príkladov sa presunieme na možnosti minimalizovania negatívnych efektov gamifikácie.

Pozitíne dôsledky gamifikácie sú bližšie rozobraté v časti ~\ref{XXXXXXXXXX}.
Negatívne dôsledky použitia gamifikácie sú blžšie rozobraté v časti ~\ref{XXXXXXXXXX} a~\ref{XXXXXXXXXX}.
Záverečné poznámky prináša časť~\ref{XXXXXXXXXX}.

\section{Čo je to gamifikácia?} \label{XXXXXXXXXX}

Tu vysvetlím pojem gamifikácia, nejaké jej vlastnosti a čo sa snaží dosiahnuť

\section{Pozitívne stránky gamifikácie} \label{XXXXXXXXXX}


Tu opíšem ako môže gamifikácia pozitívne ovplyvniť schopnosť učenia sa atď.
Táto časť bude viacej teoretická a bude sa zameriavať na vedeckú a psychyckú stránku pozitív gamifikácie

\subsection{Konkrétne prípady pozitív gamifikácie} \label{XXXXXXXXXX}

Príklady pozitív gamifikácie:

\begin{itemize}
\item XXXXXXXXX
\item XXXXXXXXX
	\begin{itemize}
	\item XXXXXXXXX
	\item XXXXXXXXX
	\end{itemize}
\end{itemize}


\section{Negatívne stránky gamifikácie} \label{XXXXXXXXXX}

Tu opíšem ako môže gamifikácia negatívne ovplyvniť schopnosť učenia sa atď.
Táto časť bude viacej teoretická a bude sa zameriavať na vedeckú a psychyckú stránku negatív gamifikácie

Môže sa mnohým zdať, že negatíva gamifikácie sú až takmer neexistujúce, ale bolo dokázané, že to tak \emph{nie je TUTO CITOVAT}~\cite{Czarnecki:Staged, Czarnecki:Progress}. Napriek tomu, aj dnes na internete narazíme na mnohé pochybné názory \cite{PLP-Framework}.


\subsection{Konkrétne prípady negatív gamifikácie} \label{XXXXXXXXXX}

Príklady negatív gamifikácie:

\begin{itemize}
\item XXXXXXXXX
\item XXXXXXXXX
	\begin{itemize}
	\item XXXXXXXXX
	\item XXXXXXXXX
	\end{itemize}
\end{itemize}


\section{Ako minimalizovať negatíva gamifikácie?} \label{XXXXXXXXXX}

\paragraph{Negatív gamifikácie sa nedá naplno zbaviť.}
To však neznamená že sa by sa jej účinky nedali minimalizovať.


\section{Záver} \label{XXXXXXXXXX} % prípadne iný variant názvu



%\acknowledgement{Ak niekomu chcete poďakovať\ldots}


% týmto sa generuje zoznam literatúry z obsahu súboru literatura.bib podľa toho, na čo sa v článku odkazujete
\bibliography{literatura}
\bibliographystyle{plain} % prípadne alpha, abbrv alebo hociktorý iný
\end{document}
