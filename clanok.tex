% Metódy inžinierskej práce

\documentclass[10pt,oneside,slovak,a4paper]{article}

\usepackage[slovak]{babel}
%\usepackage[T1]{fontenc}
\usepackage[IL2]{fontenc} % lepšia sadzba písmena Ľ než v T1
\usepackage[utf8]{inputenc}
\usepackage{graphicx}
\usepackage{url} % príkaz \url na formátovanie URL
\usepackage{hyperref} % odkazy v texte budú aktívne (pri niektorých triedach dokumentov spôsobuje posun textu)

\usepackage{cite}
%\usepackage{times}

\pagestyle{headings}

\title{Gamifikácia ako obojstranná čepeľ\thanks{Semestrálny projekt v predmete Metódy inžinierskej práce, ak. rok 2022/23, vedenie: Igor Stupavský}} 

\author{Robert Razgyel\\[2pt]
	{\small Slovenská technická univerzita v Bratislave}\\
	{\small Fakulta informatiky a informačných technológií}\\
	{\small \texttt{xrazgyel@stuba.sk}}
	}

\date{\small 6. november 2022} % upravte



\begin{document}

\maketitle

\begin{abstract}
Gamifikácia je pojem na ktorý je zameraných mnoho rôznych štúdií a článkov. Je mnohými ponímaná iba v dobrom svetle, znamenajúc, že gamifikácia je bezchybná forma vzdelávania. Nie je to ale tak, článkov a štúdií o jej negatívnej stránke nie je dostatok. Na základe tohto predpokladu sa zameriavam na otázky ''Aké sú pozitívne účinky gamifikácie?'', ''Aké sú negatívne účinky gamifikácie?'' a ''Ako minimalizovať negatíva gamifikácie?''. 
\end{abstract}



\section{Úvod}

Budeme sa venovať pozitívnym a negatívnym efektom gamifikácie. Po krátkom predstavení pojmu gamifikácia, bude uvedených niekoľkých príkladov na dobrú či zlú gamifikáciu a následne sa presunieme na možnosti minimalizovania negatívnych efektov gamifikácie.

Pojem gamifikácia je bližšie opísaný v časti ~\ref{CoJeGamifikacia}.
Pozitívne dôsledky gamifikácie sú bližšie rozobraté v časti ~\ref{PozitivnaGamifikacia} a~\ref{PozitivnaPriklady}.
Negatívne dôsledky použitia gamifikácie sú bližšie rozobraté v časti ~\ref{NegativnaGamifikacia} a~\ref{NegativnePriklady}.
Spôsob minimalizovania negatívnych efektov gamifikácie~\ref{MinimalizovanieNegatGamif}.
Záverečné poznámky prináša časť~\ref{Zaver}.

\section{Čo je to gamifikácia?} \label{CoJeGamifikacia}

Gamifikácia je istá technológia, alebo spôsob, ktorej cieľom je zaujať používateľov takým spôsobom aby vzdelávanie sa nevyzeralo ako vzdelávanie sa. V respektíve, aby sa učenie čo najviac podobalo na hranie hry. Týmto spôsobom sa mení vzťah študenta ku vzdelávaniu, zväčša k lepšiemu. Efektívna gamifikácia je taká, ktorá prijme jednotlivca aby svoju energiu, motiváciu a samotný potenciál učenia sa z hry, preniesol do učenia v reálnom svete (\emph{Gamification for Learning})~\cite{10.1007/978-3-319-97934-2_9}. Toto je podstata gamifikácia a dôvod jej popularity.

\section{Pozitívne stránky gamifikácie} \label{PozitivnaGamifikacia}

Je potrebné vedieť kedy by sa mala gamifikácia použiť. Gamifikáciu je najlepšie použiť v prípadoch, keď sú dané činnosti nudné, či až priam nepríjemné. Spôsob akým sa toto v reálnom svete dá implementovať je napríklad použitie jednoduchých odmien za každodenné činnosti. Takouto odmenou môže byť čokoľvek od sladkostí až po materiálne odmeny, ako napríklad odznak za dobre vykonanú prácu. Tento spôsob odmeňovania je často používaný v prvých ročníkoch základných škôl, kde sú deti hodnotené napríklad pečiatkou zvieratka. Je to efektívny spôsob, ako spríjemniť vypracovávanie domácich úloh. Toto funguje, lebo človek ktorý vie, že za dobre vykonanú prácu dostane odmenu. Bude sa sústreďovať na danú odmenu a nie na fakt toho že musí vykonávať činnosť ktorú nechce.

\subsection{Konkrétne prípady pozitív gamifikácie} \label{PozitivnaPriklady}
Pre ozrejmenie a poukázanie na pozitíva gamifikácie bude vymenovaných niekoľko príkladov na jej použitie.

Ako bolo spomenuté v časti ~\ref{PozitivnaGamifikacia}, odmenenie detí v školách je častá praktika. Nasledovať budú dve veľmi dôležité pozitíva gamifikácie.
\begin{itemize}
\item Motivácia pre deti do učenia, ktorá prispieva ich budúcnosti
\item Pochopenie potreby vykonávania povinností
\end{itemize}
Použitie kvízov na interaktívne vzdelávanie.
\begin{itemize}
\item Odstránenie alebo minimalizovanie vzhľadu tradičnej výučby
\item Kvôli bodom na kvízoch sa pomaly z učenia stáva súťaž o prvé miesto
\item Kvôli spoločnej hre nastáva zblíženie jednotlivých účastníkov
\end{itemize}



\section{Negatívne stránky gamifikácie} \label{NegativnaGamifikacia}
 Podľa (\emph{Gamification for Learning})~\cite{10.1007/978-3-319-97934-2_9}, zatiaľčo gamifikácia má veľký potenciál motivovať a zaujať študentov, je zároveň potrebné pochopiť, aké problémy sa pri gamifikácii nachádzajú. Mnohým sa môže zdať že negatíva gamifikácie sú až takmer neexistujúce, no nie je to tak. Podľa vyššie uvedeného zdroja nie je problémom že by gamifikácia nebola účinná. Problém je, neuvedomovanie si potenciálne neefektívnej implementácii gamifikácie, čo má za dôsledok nežiadúce negatívne účinky. 


\subsection{Konkrétne prípady negatív gamifikácie} \label{NegativnePriklady}

Príklady nesprávnych implementácií gamifikácie:
\begin{itemize}
\item Falošná gamifikácia: Pokiaľ do procesu iba zaradíme zábavné elementy, bez dlhšieho uváženia a premyslenia si taktiky implementácie, tak takáto gamifikácia negarantuje že bude efektívna. Toto je dôsledok nedostatočného pochopenia gamifikačných konceptov, návrhov, vývoja a infraštruktúry (\emph{Vrasidas a Solomou, 2013})~\cite{doi:10.1080/09523987.2013.839151}.
\item Pokiaľ je gamifikácia použitá na vzdelávanie napríklad pri vyšších vzdelaniach, ako pri študentoch vysokej školy, je potrebné dávať pozor na jej implementáciu. Pokiaľ je gamifikácia prehnaná, tak je možné že rýchlosť vzdelávania padne o značnú hodnotu.
\item Pokiaľ je gamifikácia zle implementovaná, môže sa stratiť rozdiel medzi hrou a učením. Zatiaľčo táto skutočnosť môže znieť dobre, môže mať za následok, že nás táto hra začne nudiť a keďže v hrách nie sme potrestaný, tak nebudeme ani keď sa nebudeme učiť.
\item Ak sa gamifikácia nachádza úplne všade keď sme ešte mladší, tak sa nebudeme môcť zdokonalovať v učení sa tradičným spôsobom. Toto má za následok že v dospelosti pre nás bude ťažšie sa naučiť potrebné informácie z odborných textov.
\end{itemize}




\section{Ako minimalizovať negatíva gamifikácie?} \label{MinimalizovanieNegatGamif}

\paragraph{Negatív gamifikácie sa nedá naplno zbaviť.}
To však neznamená že sa by sa jej účinky nedali minimalizovať.


\section{Záver} \label{Zaver} % prípadne iný variant názvu



%\acknowledgement{Ak niekomu chcete poďakovať\ldots}


% týmto sa generuje zoznam literatúry z obsahu súboru literatura.bib podľa toho, na čo sa v článku odkazujete
\bibliography{literatura}
\bibliographystyle{plain} % prípadne alpha, abbrv alebo hociktorý iný
\end{document}